\documentclass[article,11pt]{article}
\usepackage[a4paper,total={7in,10in}]{geometry}
\usepackage{listings}
\usepackage{xcolor}
\usepackage{tkz-euclide}
\usepackage{enumitem}
\usepackage{amsmath}
\usepackage{amssymb}
\usepackage{mathrsfs}
\usepackage[math]{cellspace}
\usepackage[french]{babel}
\setlength{\cellspacetoplimit}{3pt}
\setlength{\cellspacebottomlimit}{3pt}
\definecolor{vlightgray}{rgb}{0.9,0.9,0.9}
\usepackage[T1]{fontenc}
\lstset{
    backgroundcolor=\color{vlightgray},
    commentstyle=\color{cyan},
    stringstyle=\color{magenta},
    language=C,
    aboveskip=3mm,
    belowskip=3mm,
    literate=%
        {~}{{$\sim$}}1
        {û}{{\^u}}1
        {à}{{\`a}}1
        {é}{{\'e}}1
        {è}{{\`e}}1,
    showstringspaces=false,
    showspaces=false,
    showtabs=false,
    captionpos=b,
    keepspaces=true,
    columns=fixed,
    basicstyle=\footnotesize\ttfamily,
    numbers=right,
    numbersep=5pt,
    numberstyle=\color{purple}\tiny,
    breaklines=true,
    breakatwhitespace=false,
    tabsize=2
}

\newcommand{\eqencld}[1]{
\begin{equation*}#1\end{equation*}
}
\newcommand{\bgp}[1]{
\left(#1\right)
}
\newcommand{\bgbr}[1]{
\left\{#1\right\}
}
\newcommand{\fabs}[1]{
\left|#1\right|
}

\begin{document}

\title{Solution de l'exercice dont je connais plus le numéro}
\author{Akushiru - MP2I - Lycée Descartes}
\maketitle

\section{Développement pénible}

\begin{enumerate}
\item Visualisons les 3 cas possibles sachant que $z\neq0$ selon la base du triangle rectangle, $z$, $z^{2}$ et $z^{3}$:\\
\newcommand{\maketrec}[3]{
\begin{tikzpicture}[baseline=(current bounding box.center),xscale=-.8,yscale=-.8]
\tkzDefPoint(0,0){A} \tkzDefPoint(4,0){B}
\tkzDrawTriangle[pythagore](A,B)
\tkzGetPoint{C}
\tkzMarkRightAngle(A,B,C)
\tkzDrawPoints(A,B,C)
\tkzLabelPoint[below](A){#3}
\tkzLabelPoint[below left](B){#1}
\tkzLabelPoint[above left](C){#2}
\end{tikzpicture}
}
\begin{enumerate}[label=-]
\item Avec $z$ comme base du triangle:\maketrec{$z$}{$z^{2}$}{$z^{3}$}
\item Avec $z^{2}$ comme base du triangle: \maketrec{$z^{2}$}{$z^{3}$}{$z$}
\item Avec $z^{3}$ comme base du triangle: \maketrec{$z^{3}$}{$z$}{$z^{2}$}
\end{enumerate}
\item Pythagorisons chaque cas:
\begin{enumerate}[label=-]
\item Si le triangle est rectangle au point d'affixe $z$ alors:
\eqencld{\fabs{z^{3}-z}^{2}+\fabs{z^{2}-z}^2=\fabs{z^{3}-z^{2}}^{2}}
\eqencld{\fabs{z\bgp{z^{2}-1}}^{2}+\fabs{z\bgp{z-1}}^{2}=\fabs{z^{2}\bgp{z-1}}^{2}}
\eqencld{\fabs{z}^{2}\fabs{z-1}^{2}\fabs{z+1}^{2}+\fabs{z}^{2}\fabs{z-1}^{2}=\fabs{z^{2}}^{2}\fabs{z-1}^{2}}
\eqencld{\fabs{z}^{2}\fabs{z+1}^{2}+\fabs{z}^{2}=\fabs{z}^{4}}
\eqencld{\fabs{z+1}^{2}+1=\fabs{z}^{2}}
Avec $z=a+ib,\ \bgp{a,b}\in\mathbb{R}^{2}$:
\eqencld{\fabs{a+ib+1}^{2}+1=\fabs{a+ib}^{2}}
\eqencld{\bgp{a+1}^{2}+b^{2}+1=a^{2}+b^{2}}
\eqencld{a^{2}+2a+2=a^{2}}
\eqencld{a=-1}
\item Si le triangle est rectangle au point d'affixe $z^{2}$ alors:
\eqencld{\fabs{z-z^{2}}^{2}+\fabs{z^{2}-z^{3}}^{2}=\fabs{z-z^{3}}^{2}}
\eqencld{\fabs{1-z}^{2}+\fabs{z}^{2}\fabs{1-z}^{2}=\fabs{1+z}^{2}\fabs{1-z}^{2}}
\eqencld{1+\fabs{z}^{2}=\fabs{1+z}^{2}}
\eqencld{1+a^{2}=a^{2}+2a+1}
\eqencld{a=0} Donc $z$ est un imaginaire pur.
\item Si le triangle est rectangle au point d'affixe $z^{3}$ alors:
\eqencld{\fabs{z^{2}-z^{3}}^{2}+\fabs{z-z^{3}}^{2}=\fabs{z-z^{2}}^{2}}
\eqencld{\fabs{z}^{2}\fabs{1-z}^{2}+\fabs{1-z}^{2}\fabs{1+z}^{2}=\fabs{1-z}^{2}}
\eqencld{\fabs{z}^{2}+\fabs{1+z}^{2}=1}
\eqencld{2a^{2}+2a+2b^{2}=0}
\eqencld{a^{2}+a+b^{2}=0}
Puis on applique on légère entourloupe.
\eqencld{\bgp{a+\frac{1}{2}}^{2}-\frac{1}{4}+b^{2}=0}
\eqencld{\bgp{a+\frac{1}{2}}^{2}+b^{2}=\bgp{\frac{1}{2}}^{2}}
Bon bah ce bordel c'est une équation de cercle donc théoriquement l'ensemble des solutions c'est:
\eqencld{\mathscr{C}\bgp{\bgp{-\frac{1}{2},\ 0};\ \frac{1}{2}}}
\textsl{MAIS} il faut faire attention aux points interdits que sont $z=-1$ et $z=0$ ce qui nous donne finalement:
\eqencld{\mathscr{C}\bgp{\bgp{-\frac{1}{2},\ 0};\ \frac{1}{2}}\setminus\bgbr{\bgp{0,\ 0};\ \bgp{-1,\ 0}}}
\end{enumerate}
\end{enumerate}

\section{Conclusion}

C'est trop chiant bordel.

\end{document}