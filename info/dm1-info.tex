\documentclass[article,11pt]{article}
\usepackage[a4paper,total={7in,10in}]{geometry}
\usepackage{listings}
\usepackage{xcolor}
\usepackage{enumitem}
\usepackage{tikz}
\usetikzlibrary{calc,arrows.meta,shapes.geometric,fit,matrix,positioning}
\usepackage{amsmath}
\usepackage{bbold}
\usepackage{amssymb}
\usepackage{mathrsfs}
\usepackage{parskip}
\usepackage[math]{cellspace}
\usepackage[french]{babel}
\setlength{\cellspacetoplimit}{3pt}
\setlength{\cellspacebottomlimit}{3pt}
\definecolor{vlightgray}{rgb}{0.9,0.9,0.9}
\usepackage[T1]{fontenc}
\usepackage{minted}
\usepackage{textcomp}
\usepackage[scaled]{beramono}

\DeclareMathSymbol{\shortminus}{\mathbin}{AMSa}{"39}
\newcommand{\eqencld}[1]{
    \begin{equation*}#1\end{equation*}
}
\newcommand{\bgp}[1]{
    \left(#1\right)
}
\newcommand{\bgbr}[1]{
    \left\{#1\right\}
}
\newcommand{\fabs}[1]{
    \left|#1\right|
}

\begin{document}

\title{Devoir Maison 1 : Arbres AVL\vspace{-.75em}}
\author{Aksil Mammar}
\date{}
\maketitle

\section{Vérification du caractère AVL d'un arbre}

Dans tout l'exercice on pose $n$ le nombre de noeuds de l'arbre étudié.

\begin{enumerate}
\item Il s'agit d'un parcours infixe, on remarque que la concaténation dans la liste se fait toujours à gauche de la liste générée précédemment.\\
On pourrait donc la renommer \textsl{\textbf{parcours\_infixe}} par exemple.
\item Cette fonction a une complexité en $O(n)$, donc linéaire.
L'avantage de cette méthode vient du fait que l'on concatène des éléments au début d'une liste immuable, et donc que l'ajout de fait en temps constant.
\item Puisque le parcours infixe parcourt successivement le fils gauche, la racine puis le fils droit, il suffit de s'assurer que la liste en sortie est rangée dans l'ordre croissant.
\inputminted[fontsize=\small,firstline=7,lastline=12]{ocaml}{part1.ml}
\item La fonction précédente compare toutes les paires d'éléments successives de la liste, si cet arbre n'est pas vide alors ces paires sont au nombre de $n-1$.
Donc cet algorithme est aussi en $O(n)$.
\item On propose une fonction qui vérifie à la fois que l'arbre est équilibré et que les étiquettes sont les bonnes.
\inputminted[fontsize=\small,firstline=13,lastline=23]{ocaml}{part1.ml}
\item Utilisons les deux fonctions précédentes.
\inputminted[fontsize=\small,firstline=24,lastline=24]{ocaml}{part1.ml}
\end{enumerate}

\section{Les arbres AVL sont équilibrés}

\begin{enumerate}
\setcounter{enumi}{6}
\item Soit $n\in\mathbb{N}$, on cherche en fait à résoudre l'équation suivante: $F_{n+2}-F_{n+1}-F_{n}=0\ (E)$.\\
Supposons qu'il existe $(\lambda,\mu,r_{1},r_{2})\in\mathbb{R}^{4}$ tel que $F_{n}=\lambda r^{n}_{1}+\mu r^{n}_{2}$.\\
$(E)$ peut ainsi être reformulée grâce à la nouvelle expression:
\begin{flalign*}
&\begin{aligned}
(E)&\iff \lambda r^{n+2}_{1}+\mu r^{n+2}_{2} - \lambda r^{n+1}_{1}-\mu r^{n+1}_{2} - \lambda r^{n}_{1}-\mu r^{n}_{2}=0\\
&\iff \lambda\bgp{r^{n+2}_{1}-r^{n+1}_{1}-r^{n}_{1}}+\mu\bgp{r^{n+2}_{2}-r^{n+1}_{2}-r^{n}_{2}}=0
\end{aligned}&&
\end{flalign*}
Supposons alors que $\left\{
\begin{tabular}{@{}l@{}}
$r^{n+2}_{1}-r^{n+1}_{1}-r^{n}_{1}=0$\\
$r^{n+2}_{2}-r^{n+1}_{2}-r^{n}_{2}=0$
\end{tabular}
\right.$\\[1em]
On peut déjà expliciter $r_{1}$ et $r_{2}$ simplement, soit $r\in\mathbb{R}$:
\begin{flalign*}
&\begin{aligned}
r^{n+2}-r^{n+1}-r^{n}=0&\iff r^{2}-r-1=0\\
&\iff r=\dfrac{1\pm\sqrt{5}}{2}
\end{aligned}&&
\end{flalign*}
Donc $r_{1}=\dfrac{1+\sqrt{5}}{2}$ et $r_{2}=\dfrac{1-\sqrt{5}}{2}$, et ainsi $F_{n}=\lambda\bgp{\dfrac{1+\sqrt{5}}{2}}^{n}+\mu\bgp{\dfrac{1-\sqrt{5}}{2}}^{n}$.
Puis, il vient que:
\begin{enumerate}[label=-]
\item$F_{0}=\lambda+\mu=0\implies\mu=-\lambda$
\item$F_{1}=\lambda\dfrac{1+\sqrt{5}}{2}+\mu\dfrac{1-\sqrt{5}}{2}=\lambda\sqrt{5}=1\implies\bgp{\lambda=\dfrac{1}{\sqrt{5}}\wedge\mu=-\dfrac{1}{\sqrt{5}}}$
\end{enumerate}
Pour conclure, $\forall n\in\mathbb{N},\ F_{n}=\dfrac{1}{\sqrt{5}}\left[\bgp{\dfrac{1+\sqrt{5}}{2}}^{n}-\bgp{\dfrac{1-\sqrt{5}}{2}}^{n}\right]$.
\item Pour la preuve on pose pour tout arbre $A$ deux fonctions: $h(A)$ sa hauteur et $n(A)$ sa taille.\\[.7em]
Soit $t$ un arbre AVL supposé minimal, par induction on va montrer que:
\eqencld{n(t)=F_{h(t)+3}-1}
\textsl{\textcolor{blue}{\underline{Initialisation}:}}
\begin{enumerate}[label=-]
\item Si $h(t)=-1$ alors $n(t)=0=F_{2}-1$.
\item Si $h(t)=0$ alors $n(t)=1=F_{3}-1$.
\end{enumerate}
\textsl{\textcolor{blue}{\underline{Induction}:}}\\
On suppose désormais que $h(t)>0$, $t$ est un AVL donc il possède deux fils, un gauche $g$ et un droite $d$, au moins un des deux fils est non vide.

\begin{center}
\begin{tikzpicture}[
inner/.style={circle,draw,minimum width=6mm,inner sep=0},
level 1/.style={sibling distance=25mm}
]
\node[inner] {t}
child {
    node[inner] {g}
}
child {
    node[inner] {d}
};
\end{tikzpicture}
\end{center}
Puisque $t$ est un arbre AVL minimal, alors $\fabs{h(d)-h(g)}=1$, quitte à intervertir les rôles de $g$ et $d$ on choisit $h(g)=h(d)+1$, donc $h(t)=h(g)+1=h(d)+2$.\\
À noter que $g$ et $d$ sont minimaux, donc l'hypothèse d'induction est supposée vérifiée sur les fils de $t$.
\begin{flalign*}
&\begin{aligned}
n(t)&=n(g)+n(d)+1\\
&=F_{h(g)+3}-1+F_{h(d)+3}-1+1\hspace{1em}\text{(par hypothèse d'induction)}\\
&=F_{h(g)+3}+F_{h(d)+3}-1\\
&=F_{h(t)+2}+F_{h(t)+1}-1\\
&=F_{h(t)+3}-1\\
&=n_{h}
\end{aligned}&&
\end{flalign*}
\item Avec la proposition précédente on a que $n(t)=o(F_{h(t)+3}-1)$ donc $n(t)=o(F_{h(t)+3})$ et $F_{h(t)}=O(n(t))$.

Donc, $\dfrac{1}{\sqrt{5}}\left[\bgp{\dfrac{1+\sqrt{5}}{2}}^{h(t)}-\bgp{\dfrac{1-\sqrt{5}}{2}}^{h(t)}\right]=O(n(t))\implies\bgp{\dfrac{1+\sqrt{5}}{2}}^{h(t)}=O(n(t))$.\\
Une petite approximation donne $\phi=\dfrac{1+\sqrt{5}}{2}\approx 1{,}618034$, donc $\phi<2$.

Donc, $2^{h(t)}=O(n(t))\implies h(t)=O(\log_{2}(n(t)))$, ou plus simplement: $h=O(\log_{2}(n))$.
\item Repartons de l'expression trouvée un peu avant, cette fois-ci en gardant la constante multiplicative:
\eqencld{\dfrac{1}{\sqrt{5}}\bgp{\dfrac{1+\sqrt{5}}{2}}^{H(t)}=O(n(t))}
\eqencld{\implies\dfrac{1}{\sqrt{5}}\phi^{H(n)}=O(n),\hspace{1em}\phi=\dfrac{1+\sqrt{5}}{2}}
\eqencld{\implies H(n)=\dfrac{1}{\log_{2}(\phi)}\log_{2}(n)+c,\hspace{1em}c\in\mathbb{R}}
On trouve alors, en approximant, que $\dfrac{1}{\log_{2}(\phi)}=1{,}44042=\alpha$.
\item La hauteur d'un arbre de recherche AVL sera toujours inférieure à celle de son équivalent Rouge-Noir, ainsi on peut supposer par exemple que la recherche sera plus rapide.
\item Puisque par définition un arbre AVL est un arbre de recherche, une méthode plus adaptée est possible.
\inputminted[fontsize=\small,firstline=25,lastline=27]{ocaml}{part2.ml}
\item Dans le pire cas l'élément recherché n'est pas dans l'arbre, dans ce cas l'algorithme parcourt tout l'arbre jusqu'à atteindre une feuille.\\
En supposant que cette feuille est au dernier étage il faut parcourir $h$ noeuds, donc l'algorithme est en $O(log_{2}(n))$.
\item On peut exploiter la variable de déséquilibre, celle-ci indique quel sous-arbre est prépondérant et on peut donc omettre une branche entière pendant le parcours.
\inputminted[fontsize=\small,firstline=28,lastline=30]{ocaml}{part2.ml}
\end{enumerate}

\section{Rééquilibrage par rotations}

\begin{enumerate}
\setcounter{enumi}{14}
\item Observons l'état de l'arbre après une rotation droite.
\begin{center}
\begin{tikzpicture}[
inner/.style={circle,draw,minimum width=7mm,inner sep=0},
leaf/.style={isosceles triangle,draw,shape border rotate=90,isosceles triangle stretches=true, minimum height=10mm,minimum width=12mm,inner sep=0,yshift={-5mm},font=\footnotesize},
large leaf/.style={leaf,minimum height=30mm,yshift={-13.4mm}},
level 1/.style={sibling distance=30mm},
level 2/.style={sibling distance=21mm},
level 3/.style={sibling distance=14mm},
]
\node[inner] {$y$}
[child anchor=north]
child {
    node[large leaf] {$A$}
}
child {
    node[inner] {$x$}
    child {
        node[large leaf] {$B$}
    }
    child {
        node[leaf] {$C$}
    }
};
\draw[dotted,line width=0.3mm] (-3.25,-3.81) -- (4.25,-3.81) node[right] {$1$};
\draw[dotted,line width=0.3mm] (-3.25,-5.3) -- (4.25,-5.3) node[right] {$2$};
\end{tikzpicture}
\end{center}
Le sous-arbre $C$ gagne un niveau, le sous-arbre $A$ en perd un et le $B$ ne change pas de niveau. Et puisqu'une rotation conserve l'arrangement d'un ABR, la qualité AVL est restaurée.
\item À nouveau, observons l'arbre initial puis son état après la même procédure de rotation.
\begin{center}
\begin{tikzpicture}[
inner/.style={circle,draw,minimum width=7mm,inner sep=0},
leaf/.style={isosceles triangle,draw,shape border rotate=90,isosceles triangle stretches=true, minimum height=10mm,minimum width=12mm,inner sep=0,yshift={-5mm},font=\footnotesize},
large leaf/.style={leaf,minimum height=30mm,yshift={-13.4mm}},
level 1/.style={sibling distance=30mm},
level 2/.style={sibling distance=21mm},
level 3/.style={sibling distance=14mm},
]
\node[inner] {$x$}
[child anchor=north]
child {
    node[inner] {$y$}
    child {
        node[large leaf] {$A$}
    }
    child {
        node[leaf] {$B$}
    }
}
child {
    node[leaf] {$C$}
};
\draw[dotted,line width=0.3mm] (-3.25,-2.3) -- (4.25,-2.3) node[right] {$0$};
\draw[dotted,line width=0.3mm] (-3.25,-3.81) -- (4.25,-3.81) node[right] {$1$};
\draw[dotted,line width=0.3mm] (-3.25,-5.3) -- (4.25,-5.3) node[right] {$2$};
\end{tikzpicture}
\end{center}
Puis on effectue la rotation:
\begin{center}
\begin{tikzpicture}[
inner/.style={circle,draw,minimum width=7mm,inner sep=0},
leaf/.style={isosceles triangle,draw,shape border rotate=90,isosceles triangle stretches=true, minimum height=10mm,minimum width=12mm,inner sep=0,yshift={-5mm},font=\footnotesize},
large leaf/.style={leaf,minimum height=30mm,yshift={-13.4mm}},
level 1/.style={sibling distance=30mm},
level 2/.style={sibling distance=21mm},
level 3/.style={sibling distance=14mm},
]
\node[inner] {$y$}
[child anchor=north]
child {
    node[large leaf] {$A$}
}
child {
    node[inner] {$x$}
    child {
        node[leaf] {$B$}
    }
    child {
        node[leaf] {$C$}
    }
};
\draw[dotted,line width=0.3mm] (-3.25,-3.81) -- (4.25,-3.81) node[right] {$1$};
\end{tikzpicture}
\end{center}
\item\begin{enumerate}[label=\alph*)]
\item Supposons que le sous-arbre $D$ a une hauteur $h_{D}$, $x$ a un déséquilibre de $-2$ donc $y$ a une hauteur notée $h_{y}=h_{D}+2$. $y$ a lui-même un déséquilibre de $+1$ donc
$h_{y}=h_{z}+1$, ainsi on obtient que $h_{z}=h_{D}+1$, et donc nécessairement $h_{A}=h_{D}$.

On se retrouve bien dans la situation exposée dans l'énoncé. Par ailleurs, tant que le sous-arbre $z$ respecte les conditions précédentes, son déséquilibre n'importe pas.
\item L'arbre peut être rééquilibré par une rotation gauche:
\begin{center}
\begin{tikzpicture}[
inner/.style={circle,draw,minimum width=7mm,inner sep=0},
leaf/.style={isosceles triangle,draw,shape border rotate=90,isosceles triangle stretches=true, minimum height=10mm,minimum width=12mm,inner sep=0,yshift={-5mm},font=\footnotesize},
large leaf/.style={leaf,minimum height=30mm,yshift={-13.4mm}},
level 1/.style={sibling distance=30mm},
level 2/.style={sibling distance=21mm},
level 3/.style={sibling distance=14mm},
]
\node[inner] {$x$}
[child anchor=north]
child {
    node[inner] {$z$}
    child {
        node[inner] {$y$}
        child {
            node[leaf] {$A$}
        }
        child {
            node[large leaf] {$B$}
        }
    }
    child {
        node[leaf] {$C$}
    }
}
child {
    node[leaf] {$D$}
};
\draw[dotted,line width=0.3mm] (-4.25,-2.31) -- (4.25,-2.31) node[right] {$0$};
\draw[dotted,line width=0.3mm] (-4.25,-3.81) -- (4.25,-3.81) node[right] {$1$};
\draw[dotted,line width=0.3mm] (-4.25,-5.31) -- (4.25,-5.31) node[right] {$2$};
\draw[dotted,line width=0.3mm] (-4.25,-6.81) -- (4.25,-6.81) node[right] {$3$};
\end{tikzpicture}
\end{center}
Puis par une rotation droite sur la racine:
\begin{center}
\begin{tikzpicture}[
inner/.style={circle,draw,minimum width=7mm,inner sep=0},
leaf/.style={isosceles triangle,draw,shape border rotate=90,isosceles triangle stretches=true, minimum height=10mm,minimum width=12mm,inner sep=0,yshift={-5mm},font=\footnotesize},
large leaf/.style={leaf,minimum height=30mm,yshift={-13.4mm}},
level 1/.style={sibling distance=30mm},
level 2/.style={sibling distance=17mm},
level 3/.style={sibling distance=14mm},
]
\node[inner] {$z$}
[child anchor=north]
child {
    node[inner] {$y$}
    child {
        node[leaf] {$A$}
    }
    child {
        node[large leaf] {$B$}
    }
}
child {
    node[inner] {$x$}
    child {
        node[leaf] {$C$}
    }
    child {
        node[leaf] {$D$}
    }
};
\draw[dotted,line width=0.3mm] (-4.25,-2.31) -- (4.25,-2.31) node[right] {$0$};
\draw[dotted,line width=0.3mm] (-4.25,-3.81) -- (4.25,-3.81) node[right] {$1$};
\draw[dotted,line width=0.3mm] (-4.25,-5.31) -- (4.25,-5.31) node[right] {$2$};
\end{tikzpicture}
\end{center}
\item\begin{enumerate}[label=-]
\item$\delta^{'}_{x}=\begin{cases}
1 & \text{si }\delta^{'}_{z}=-1\\
0 & \text{sinon}
\end{cases}$
\item$\delta^{'}_{y}=\begin{cases}
-1 & \text{si }\delta^{'}_{z}=1\\
0 & \text{sinon}
\end{cases}$
\item$\delta^{'}_{z}=0$
\end{enumerate}
\end{enumerate}
\item On note $h$ la hauteur initiale et $h^{'}$ la hauteur après rééquilibrage.
\begin{enumerate}[label=\arabic*)]
\item Si $\delta_{y}=0$ alors $h^{'}=h$.
\item Si $\delta_{y}=-1$ alors $h^{'}=h-1$.
\item Si $\delta_{y}=+1$ et $\delta_{z}\in\bgbr{-1,0,+1}$ alors $h^{'}=h-1$.
\end{enumerate}
\end{enumerate}

\section{Insertion dans un arbre AVL}

\begin{enumerate}
\setcounter{enumi}{18}
\item Voici l'arbre après insertion de $22$ avec une rotation droite sur le noeud $25$.
\begin{center}
\begin{tikzpicture}[
inner/.style={circle,draw,minimum width=7mm,inner sep=0},
level 1/.style={sibling distance=40mm},
level 2/.style={sibling distance=20mm},
level 3/.style={sibling distance=30mm},
level 4/.style={sibling distance=15mm}
]
\node[inner] {$30$}
child {
    node[inner] {$20$}
    child {
        node[inner] {$10$}
        child {
            node[inner] {$3$}
        }
        child[missing]
    }
    child {
        node[inner] {$23$}
        child {
            node[inner] {$21$}
            child[missing]
            child {
                node[inner] {$22$}
            }
        }
        child {
            node[inner] {$25$}
            child {
                node[inner] {$24$}
            }
            child {
                node[inner] {$29$}
            }
        }
    }
}
child {
    node[inner] {$40$}
    child {
        node[inner] {$35$}
        child[missing]
        child {
            node[inner] {$37$}
        }
    }
    child {
        node[inner] {$50$}
    }
};
\end{tikzpicture}
\end{center}
Puis après l'insertion de $26$ avec une rotation droite sur le noeud $20$.
\begin{center}
\begin{tikzpicture}[
inner/.style={circle,draw,minimum width=7mm,inner sep=0},
level 1/.style={sibling distance=45mm},
level 2/.style={sibling distance=30mm},
level 3/.style={sibling distance=15mm},
level 4/.style={sibling distance=15mm}
]
\node[inner] {$30$}
child {
    node[inner] {$23$}
    child {
        node[inner] {$20$}
        child {
            node[inner] {$10$}
            child {
                node[inner] {$3$}
            }
            child[missing]
        }
        child {
            node[inner] {$21$}
            child[missing]
            child {
                node[inner] {$22$}
            }
        }
    }
    child {
        node[inner] {$25$}
        child {
            node[inner] {$24$}
        }
        child {
            node[inner] {$29$}
            child {
                node[inner] {$26$}
            }
            child[missing]
        }
    }
}
child {
    node[inner] {$40$}
    child {
        node[inner] {$35$}
        child[missing]
        child {
            node[inner] {$37$}
        }
    }
    child {
        node[inner] {$50$}
    }
};
\end{tikzpicture}
\end{center}
Et enfin après insertion de $27$ avec deux rotations (gauche-droite) sur le noeud $29$.
\begin{center}
\begin{tikzpicture}[
inner/.style={circle,draw,minimum width=7mm,inner sep=0},
level 1/.style={sibling distance=45mm},
level 2/.style={sibling distance=30mm},
level 3/.style={sibling distance=15mm},
level 4/.style={sibling distance=15mm}
]
\node[inner] {$30$}
child {
    node[inner] {$23$}
    child {
        node[inner] {$20$}
        child {
            node[inner] {$10$}
            child {
                node[inner] {$3$}
            }
            child[missing]
        }
        child {
            node[inner] {$21$}
            child[missing]
            child {
                node[inner] {$22$}
            }
        }
    }
    child {
        node[inner] {$25$}
        child {
            node[inner] {$24$}
        }
        child {
            node[inner] {$27$}
            child {
                node[inner] {$26$}
            }
            child {
                node[inner] {$29$}
            }
        }
    }
}
child {
    node[inner] {$40$}
    child {
        node[inner] {$35$}
        child[missing]
        child {
            node[inner] {$37$}
        }
    }
    child {
        node[inner] {$50$}
    }
};
\end{tikzpicture}
\end{center}
\item Posons l'hypothèse d'induction suivante pour un arbre AVL de taille $n$:
\eqencld{H(n):\ \text{\guillemotleft Après insertion d'un noeud, la hauteur reste la même ou est incrémentée.\guillemotright}}
Pour la preuve on note $h$ la hauteur de l'arbre avant insertion et $h^{'}$ la hauteur de l'arbre après insertion.

\textsl{\textcolor{blue}{\underline{Initialisation}:}}\\
Les cas $n=0$ et $n=1$ sont triviaux car on a toujours $h^{'}=h+1$.

\textsl{\textcolor{blue}{\underline{Induction}:}}\\
Supposons que $n>1$, il y a deux cas à considérer:
\begin{enumerate}[label=-]
\item Après insertion, l'arbre est équilibré. Alors aucune rotation n'est nécessaire, par induction les deux sous-arbres sont de taille $h$ ou $h-1$, donc $h^{'}\in\bgbr{h,h+1}$.
\item Après insertion, l'arbre est déséquilibré. Alors il faut effectuer une ou plusieurs rotations. Puisque l'arbre est désormais équilibré on revient au cas précédent.
\end{enumerate}
\item Les cas $2$ et $5$ ne sont pas possibles.
\item Après insertion, et même après les rotations, les noeuds parents gardent leurs facteurs de déséquilibre, donc l'insertion initiale est la seule contrainte en temps.
\item Sachant que la hauteur d'un arbre AVL est en $\log(n)$ alors l'insertion est en compléxité logarithmique également.
\item On couvre les cinq cas, ce qui donne:
\inputminted[fontsize=\small,firstline=31,lastline=37]{ocaml}{part4.ml}
\item Il suffit de permuter les fils, et la procédure est la même:
\inputminted[fontsize=\small,firstline=38,lastline=44]{ocaml}{part4.ml}
\end{enumerate}

\end{document}