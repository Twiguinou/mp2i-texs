\documentclass[article,11pt]{article}
\usepackage[a4paper,total={7in,10in}]{geometry}
\usepackage{listings}
\usepackage{xcolor}
\usepackage[europeanresistors]{circuitikz}
\usepackage{enumitem}
\usepackage{amsmath}
\usepackage{pgfplots}
\usepackage{siunitx}
\sisetup{output-decimal-marker={,},locale=FR}
\usepackage{amssymb}
\usepackage{mathrsfs}
\usepackage{parskip}
\usepackage{accents}
\usepackage[math]{cellspace}
\usepackage[french]{babel}
\setlength{\cellspacetoplimit}{3pt}
\setlength{\cellspacebottomlimit}{3pt}
\definecolor{vlightgray}{rgb}{0.9,0.9,0.9}
\usepackage[T1]{fontenc}
\lstset{
    backgroundcolor=\color{vlightgray},
    commentstyle=\color{cyan},
    stringstyle=\color{magenta},
    language=C,
    aboveskip=3mm,
    belowskip=3mm,
    literate=%
        {~}{{$\sim$}}1
        {û}{{\^u}}1
        {à}{{\`a}}1
        {é}{{\'e}}1
        {è}{{\`e}}1,
    showstringspaces=false,
    showspaces=false,
    showtabs=false,
    captionpos=b,
    keepspaces=true,
    columns=fixed,
    basicstyle=\footnotesize\ttfamily,
    numbers=right,
    numbersep=5pt,
    numberstyle=\color{purple}\tiny,
    breaklines=true,
    breakatwhitespace=false,
    tabsize=2
}

\newcommand{\eqencld}[1]{
    \begin{equation*}#1\end{equation*}
}
\newcommand{\bgp}[1]{
    \left(#1\right)
}
\newcommand{\bgbr}[1]{
    \left\{#1\right\}
}
\newcommand{\fabs}[1]{
    \left|#1\right|
}

\begin{document}

\textbf{\NoAutoSpacing Exercice 1:} Étude d'un filtre RL
\begin{enumerate}
\item\noindent\begin{enumerate}[label=-]
\item À basse fréquence la bobine est équivalente à un fil.
\item À haute fréquence la bobine est équivalente à un interrupteur ouvert, ainsi la tension $u_{s}$ est nulle.
\end{enumerate}
\item La formule pour la fonction de transfert est $H=\dfrac{u_{s}}{u_{e}}$, en impédance complexe $\underaccent{\bar}{H}=\dfrac{\underaccent{\bar}{u}_{s}}{\underaccent{\bar}{u}_{e}}$.
En utilisant un pont diviseur de tension on obtient;
\begin{flalign*}
&\begin{aligned}
\underaccent{\bar}{H}&=\dfrac{1}{\underaccent{\bar}{u}_{e}}\cdot{\underaccent{\bar}{u}_{e}}\cdot\dfrac{z_{R}}{z_{R}+z_{L}}\\
&=\dfrac{R}{R+jL\omega}\\
&=\dfrac{1}{1+j\frac{L}{R}\omega}\quad\text{on pose }\left\{\begin{aligned}
&H_{0}=1\\
&\omega_{c}=\dfrac{R}{L}
\end{aligned}\right.\quad\text{ou bien }\left\{\begin{aligned}
&H_{0}=1\\
&x=\dfrac{L}{R}\omega
\end{aligned}\right.\quad\text{pour la forme canonique j'ai pas bien compris.}\\
&=\dfrac{1}{1+j\frac{\omega}{\omega_{c}}}
\end{aligned}&&
\end{flalign*}
Puis on peut déterminer le gain $H(\omega)$:
\begin{flalign*}
&\begin{aligned}
H(\omega)&=\fabs{\underaccent{\bar}{H}}\\
&=\fabs{\dfrac{1}{1+j\frac{\omega}{\omega_{c}}}}\\
&=\dfrac{1}{\fabs{1+j\frac{\omega}{\omega_{c}}}}\\
&=\dfrac{1}{\sqrt{1+\bgp{\frac{\omega}{\omega_{c}}}^{2}}}
\end{aligned}&&
\end{flalign*}
\begin{tabular}{|c|c|c|c|}
\hline $x=\frac{\omega}{\omega_{c}}$ & $x\xrightarrow[\text{\tiny (B.F)}]{} 0$ & $x\xrightarrow[\text{\tiny (H.F)}]{}+\infty$ & $x=1$\\
\hline $\underaccent{\bar}{H}$ & $1$ & $0$ & $\frac{1}{\sqrt{2}}e^{-i\frac{\pi}{4}}$\\
\hline $H(\omega)$ & $1$ & $0$ & $\frac{1}{\sqrt{2}}$\\
\hline $\varPhi_{s/e}=\arg(\underaccent{\bar}{H})$ & $0$ & $-\frac{\pi}{2}$ & $-\frac{\pi}{4}$\\
\hline $G_{\SI{}{\decibel}}=20\log(H)$ & $0$ & $0$ & $20\cdot\log(H_{\max})-\SI{3}{\decibel}$\\
\hline
\end{tabular}\\[2em]
\item
\end{enumerate}

\end{document}