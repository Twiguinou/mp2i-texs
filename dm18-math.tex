\documentclass[article,11pt]{article}
\usepackage[a4paper,total={7in,10in}]{geometry}
\usepackage{listings}
\usepackage{xcolor}
\usepackage{enumitem}
\usepackage{tikz}
\usetikzlibrary{calc,arrows.meta,shapes.geometric,fit,matrix,positioning}
\usepackage{amsmath}
\usepackage{bbold}
\usepackage{amssymb}
\usepackage{mathrsfs}
\usepackage{parskip}
\usepackage[math]{cellspace}
\usepackage[french]{babel}
\setlength{\cellspacetoplimit}{3pt}
\setlength{\cellspacebottomlimit}{3pt}
\definecolor{vlightgray}{rgb}{0.9,0.9,0.9}
\usepackage[T1]{fontenc}

\DeclareMathSymbol{\shortminus}{\mathbin}{AMSa}{"39}
\newcommand{\eqencld}[1]{
    \begin{equation*}#1\end{equation*}
}
\newcommand{\bgp}[1]{
    \left(#1\right)
}
\newcommand{\bgbr}[1]{
    \left\{#1\right\}
}
\newcommand{\fabs}[1]{
    \left|#1\right|
}

\begin{document}

\title{Devoir Maison 18 : Sous-espaces vectoriels\vspace{-.75em}}
\author{Aksil Mammar}
\date{}
\maketitle

\section{Introduction}

\begin{enumerate}
\item\begin{enumerate}
\item Puisque $A$ et $B$ sont des sous-espaces vectoriels de $E$ alors $A\cap B$ également.
Et $A\cap B\subset B$ donc il existe $C\subset B$ un sous-espace vectoriel de $E$ tel que $C\oplus\bgp{A\cap B}=B$.

Procédons ensuite par double inclusion:
\begin{enumerate}[label=-]
\item Soit $u\in A\oplus C,\ \exists!(v,w)\in A\times C,\ u=v+w$, et $C\subset B$ donc $w\in B$, donc $u\in A+B$.
\item Soit $u\in A+B,\ \exists(v,w)\in A\times B,\ u=v+w$, on sait que $C\oplus\bgp{A\cap B}=B$ donc $\exists!(x,y)\in C\times\bgp{A\cap B},\ w=x+y$.\\
Ainsi, $u=v+x+y=(v+y)+x$, avec $v+y\in A$ et $y\in C$, donc $u\in A\oplus C$.
\end{enumerate}
Donc, par double inclusion, $A\oplus C=A+B$.
\item Développons l'expression suivante:
\begin{flalign*}
&\begin{aligned}
\dim\bgp{A+B}&=\dim\bgp{A\oplus C}\\
&=\dim\bgp{A}+\dim\bgp{C}
\end{aligned}&&
\end{flalign*}
Or on sait que $C\oplus\bgp{A\cap B}=B$ donc $\dim\bgp{C}=\dim\bgp{B}-\dim\bgp{A\cap B}$.
\eqencld{\dim\bgp{A+B}=\dim\bgp{A}+\dim\bgp{B}-\dim\bgp{A\cap B}}
\end{enumerate}
\item\begin{enumerate}
\item Le vecteur $x$ n'est pas dans $A$ donc il ne peut pas être écrit comme une combinaison linéaire des vecteurs $A$.\\
Posons $\dim\bgp{A}=n$, $\bgp{a_{1},...,a_{n}}$ une base de $A$, on peut alors écrire $\bgp{a_{1},...,a_{n},x}$ une base de $\mathrm{Vect}\bgp{A\cup\bgbr{x}}$.

Ainsi, tout vecteur de $\mathrm{Vect}\bgp{A\cup\bgbr{x}}$ est une combinaison linéaire, unique, des $\bgbr{a_{1},...,a_{n}}$ et de $x$.
Donc on en déduit que:
\begin{flalign*}
&\begin{aligned}
\mathrm{Vect}\bgp{A\cup\bgbr{x}}&=\bgbr{\lambda_{1}a_{1}+...+\lambda_{n}a_{n}+\lambda x,\ \bgp{\lambda_{1},...,\lambda_{n},\lambda}\in\mathbb{K}^{n+1}}\\
&=\bgbr{a+\lambda x,\ \bgp{a,\lambda}\in A\times\mathbb{K}}
\end{aligned}&&
\end{flalign*}
\item$\bgp{a_{1},...,a_{n},x}$ est une base de $\mathrm{Vect}\bgp{A\cup\bgbr{x}}$ de dimension $n+1=\dim\bgp{A}+1$.
\end{enumerate}
\item Soit $n=\dim\bgp{E}$, on ne se préoccupe pas du cas $n=0$, alors puisque $A$ et $B$ sont des hyperplans de $E$, $\dim\bgp{A}=\dim\bgp{B}=n-1$.\\
Avec la formule de Grassman, $\dim\bgp{A\cap B}=2n-2-\dim\bgp{A+B}$.

Aussi $A+B=E$ car $A$ et $B$ sont distincts, donc $\dim\bgp{A+B}=\dim\bgp{E}=n$, donc en reformulant:
\eqencld{\dim\bgp{A\cap B}=n-2=\dim\bgp{B}-1}
Et, puisque $A\cap B\subset B$, alors $A\cap B$ est un hyperplan de $B$.
\item Posons avant tout $n=\dim\bgp{E}$, il y a trois cas à considérer, on suppose que $n>0$.
\begin{enumerate}[label=-]
\item Si $\dim\bgp{A}=n-1$ alors $A$ est un hyperplan donc c'est bon.
\item Si $\dim\bgp{A}=k<n-1$, soit $\bgp{a_{1},...,a_{k}}$ une base de $A$, on peut la compléter de $n-k-1$ vecteurs pour former un hyperplan de $E$, qui contient donc $A$.
\end{enumerate}
\end{enumerate}

\end{document}