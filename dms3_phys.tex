\documentclass[article,11pt]{article}
\usepackage[a4paper,total={7in,10in}]{geometry}
\usepackage{listings}
\usepackage{xcolor}
\usepackage[europeanresistors]{circuitikz}
\usepackage{enumitem}
\usepackage{amsmath}
\usepackage{pgfplots}
\usepackage{siunitx}
\sisetup{output-decimal-marker={,},locale=FR}
\usepackage{amssymb}
\usepackage{mathrsfs}
\usepackage[math]{cellspace}
\usepackage[french]{babel}
\setlength{\cellspacetoplimit}{3pt}
\setlength{\cellspacebottomlimit}{3pt}
\definecolor{vlightgray}{rgb}{0.9,0.9,0.9}
\usepackage[T1]{fontenc}
\lstset{
    backgroundcolor=\color{vlightgray},
    commentstyle=\color{cyan},
    stringstyle=\color{magenta},
    language=C,
    aboveskip=3mm,
    belowskip=3mm,
    literate=%
        {~}{{$\sim$}}1
        {û}{{\^u}}1
        {à}{{\`a}}1
        {é}{{\'e}}1
        {è}{{\`e}}1,
    showstringspaces=false,
    showspaces=false,
    showtabs=false,
    captionpos=b,
    keepspaces=true,
    columns=fixed,
    basicstyle=\footnotesize\ttfamily,
    numbers=right,
    numbersep=5pt,
    numberstyle=\color{purple}\tiny,
    breaklines=true,
    breakatwhitespace=false,
    tabsize=2
}

\newcommand{\eqencld}[1]{
    \begin{equation*}#1\end{equation*}
}
\newcommand{\bgp}[1]{
    \left(#1\right)
}
\newcommand{\bgbr}[1]{
    \left\{#1\right\}
}
\newcommand{\fabs}[1]{
    \left|#1\right|
}

\begin{document}

\title{Devoir Maison 5: Exercice 3 - Signal apériodique}
\author{Aksil Mammar - MP2I - Lycée Descartes}
\date{}
\maketitle

\begin{enumerate}
\item En $t=0^{+}$ on a $U_{C}\bgp{0^{+}}=U_{C}\bgp{0^{-}}=\SI{1}{\volt}$ et $i\bgp{0^{+}}=C\dfrac{dU_{C}}{dt}\bgp{0^{-}}=\SI{0}{\ampere}$ par principe de continuité du condensateur.
\item En régime permanent, c'est-à-dire en $t\longrightarrow+\infty$, $U_{C}\bgp{+\infty}=0$ et $i\bgp{+\infty}=C\dfrac{dU_{C}}{dt}=0$
\item La loi des mailles nous donne l'équation suivante.
\eqencld{U_{C}+U_{R}+U_{L}=0}
\eqencld{\dfrac{q}{C}+Ri\bgp{t}+L\dfrac{di}{dt}\bgp{t}=0}
\eqencld{\dfrac{1}{C}i\bgp{t}+R\dfrac{di}{dt}\bgp{t}+L\dfrac{d^{2}i}{dt^{2}}\bgp{t}=0}
\eqencld{\dfrac{d^{2}i}{dt^{2}}\bgp{t}+\dfrac{R}{L}\dfrac{di}{dt}\bgp{t}+\dfrac{1}{LC}i\bgp{t}=0}
Puis on considère $\omega_{0}=\dfrac{1}{\sqrt{LC}}$, $Q=\dfrac{1}{R}\sqrt{\dfrac{L}{C}}$ et $\lambda=\dfrac{\omega_{0}}{2Q}$:
\eqencld{\dfrac{d^{2}i}{dt^{2}}\bgp{t}+2\lambda\dfrac{di}{dt}\bgp{t}+\omega^{2}_{0}i\bgp{t}=0\ \bgp{E}}
On calcule $\lambda$ et $\omega_{0}$ à l'aide de $R=\SI{6000}{\ohm}$, $C=\SI{1e-7}{\farad}$ et $L=\SI{0.0015}{\henry}$:\\
$\omega_{0}=\dfrac{1}{\sqrt{LC}}\approx\SI{81649{,}65}{\radian\per\second}$, $Q\approx0{,}0204$ et $\lambda=\dfrac{\omega_{0}}{2Q}=\dfrac{R}{2L}=\SI{2e6}{\per\second}$.\\
Puis l'équation caractéristique de $\bgp{E}$ est la suivante:
\eqencld{\alpha^{2}+2\lambda\alpha+\omega^{2}_{0}=0\ \bgp{E_{c}}}
Dont le discrimant est $\Delta=4\lambda^{2}-4\omega^{2}_{0},\ \sqrt{\Delta}=2\sqrt{\lambda^{2}-\omega^{2}_{0}}$, on remarque ainsi que, puisque
$\lambda>\omega_{0}$, la racine du discriminant est réelle.
\eqencld{\lambda>\omega_{0}\iff\dfrac{\omega_{0}}{2Q}>\omega_{0}\iff Q<\cfrac{1}{2}}
Le régime est donc apériodique puisque $Q=0{,}0204$ satisfait la condition précédente.
\item Les deux solutions $\alpha_{1}$ et $\alpha_{2}$ de $\bgp{E_{c}}$ s'expriment par:
\eqencld{\alpha_{1}=-\lambda+\sqrt{\lambda^{2}-\omega^{2}_{0}}\ \text{ et }\ \alpha_{2}=-\lambda-\sqrt{\lambda^{2}-\omega^{2}_{0}}}
Donc la solution générale de $\bgp{E}$ est:
\begin{flalign*}
\begin{aligned}
i\bgp{t}&=Ae^{\alpha_{1}t}+Be^{\alpha_{2}t},\ \bgp{A,B}\in\mathbb{R}^{2}\\
&=Ae^{\bgp{-\lambda+\sqrt{\lambda^{2}-\omega^{2}_{0}}}t}+Be^{\bgp{-\lambda-\sqrt{\lambda^{2}-\omega^{2}_{0}}}t}
\end{aligned}
\end{flalign*}
Pour des valeurs réalistes de $A$ et $B$, la courbe serait la suivante:\\[2em]
\pgfmathsetseed{\number\randomseed}
\begin{tikzpicture}
\pgfmathsetmacro{\fsola}{-3.1}
\pgfmathsetmacro{\fsolb}{-2.1}
\pgfmathsetmacro{\constantA}{0.5}
\begin{axis}[xlabel={$t$},ylabel={$i\bgp{t}$},ticks=none,domain=0:9]
\addplot[mark=none,samples=200]{\constantA * exp(\fsola * x) - \constantA * exp(\fsolb * x)};
\end{axis}
\end{tikzpicture}
\item On sait que $i\bgp{0}=A+B=0$ donc $B=-A$.
\eqencld{i\bgp{t}=A\bgp{e^{\bgp{-\lambda+\sqrt{\lambda^{2}-\omega^{2}_{0}}}t}-e^{\bgp{-\lambda-\sqrt{\lambda^{2}-\omega^{2}_{0}}}t}},\ A\in\mathbb{R}}
Or avec l'inductance: $L\dfrac{di}{dt}\bgp{0}=-U_{0}\iff\dfrac{di}{dt}\bgp{0}=-\dfrac{U_{0}}{L}$
\eqencld{\dfrac{di}{dt}\bgp{t}=A\bgp{\bgp{-\lambda+\sqrt{\lambda^{2}-\omega^{2}_{0}}}e^{\bgp{-\lambda+\sqrt{\lambda^{2}-\omega^{2}_{0}}}t}-\bgp{-\lambda-\sqrt{\lambda^{2}-\omega^{2}_{0}}}e^{\bgp{-\lambda-\sqrt{\lambda^{2}-\omega^{2}_{0}}}t}}}
Donc les conditions initiales nous donnent:
\begin{flalign*}
&\begin{aligned}
\dfrac{di}{dt}\bgp{0}&=A\bgp{\bgp{-\lambda+\sqrt{\lambda^{2}-\omega^{2}_{0}}}+\bgp{\lambda+\sqrt{\lambda^{2}-\omega^{2}_{0}}}}\\
&=2A\sqrt{\lambda^{2}-\omega^{2}_{0}}=-\dfrac{U_{0}}{L}
\end{aligned}&&
\end{flalign*}
\eqencld{A=-\dfrac{U_{0}}{2L\sqrt{\lambda^{2}-\omega^{2}_{0}}}}
D'où l'expression finale de $i$:
\eqencld{i\bgp{t}=\dfrac{U_{0}}{2L\sqrt{\lambda^{2}-\omega^{2}_{0}}}\bgp{e^{\bgp{-\lambda+\sqrt{\lambda^{2}-\omega^{2}_{0}}}t}-e^{\bgp{-\lambda-\sqrt{\lambda^{2}-\omega^{2}_{0}}}t}}}
\end{enumerate}

\end{document}